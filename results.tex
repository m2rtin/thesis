\chapter{Results}


\section{MTA Contest}

přípspěvek do soutěže at: \\
\url{http://challengepost.com/software/alex-information-about-public-transportation-in-new-york}
stránky at:\\
\url{http://alex-ptien.com/}

button and how it is accessible via mobile device with web browser supporting webrtc(google chrome)

\section{CrowdFlower - subjective user satisfaction}

\todo{do výsledků říct kolik jobů jsme pustili a po jakých dávkách}


\subsection{Google ASR}
\subsection{Kaldi ASR}


\section{Future work}

\todo{to implement MTA API which would give us the ability to support more accurate information about current connections, locations of trains etc.}


\todo{implement MTA live data support which could be much more complexly used, to be able to ask how many minutes is the next train due our initial station would be cool}

% udělat, aby to rozumnělo na rohu páté a deváté -> vyinferovat street/avenue, nebo se i zeptat. Lidi ale nebyli v komunikaci tak familierní, aby používali takovýhle hantec, řikali to postupně a oficiálně, jakoby to zadávaly do vyhledávače. Což se zdá jen otázka času, kdy si lidi zvyknou a bude jim to přirozenější bavit se s počítačem jako se svým kámošem.


PTICS is focused on providing information relevant to prague integration transport. but we have to be more flexible than this. We wanted to support intersections as výchozí body, that is from fifth street and twenty second street or vice versa. We ended up supporting not only intersections but plain streets as valid input. This way one can go from fifth avenue even though it is not really an apt location.



\todo{Develop statistical SLU for robustness}


\section{Acknowledgements}

Access to computing and storage facilities owned by parties and projects contributing to the National Grid Infrastructure MetaCentrum, provided under the programme ``Projects of Large Infrastructure for Research, Development, and Innovations'' (LM2010005), is greatly appreciated.