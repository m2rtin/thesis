\chapter{Results}


\section{MTA Contest}

\section{Google ASR}
\section{Kaldi ASR}

\section{Future work}

\todo{to implement MTA API which would give us the ability to support more accurate information about current connections, locations of trains etc.}

udělat, aby to rozumnělo na rohu páté a deváté -> vyinferovat street/avenue, nebo se i zeptat. Lidi ale nebyli v komunikaci tak familierní, aby používali takovýhle hantec, řikali to postupně a oficiálně, jakoby to zadávaly do vyhledávače. Což se zdá jen otázka času, kdy si lidi zvyknou a bude jim to přirozenější bavit se s počítačem jako se svým kámošem.


PTICS is focused on providing information relevant to prague integration transport. but we have to be more flexible than this. We wanted to support intersections as výchozí body, tzn. from fifth street and twenty second street nebo naopak. We ended up supporting not only intersections but samotný streets as valid input. This way one can go from fifth avenue even though it is not really 



\todo{Develop statistical SLU for robustness}