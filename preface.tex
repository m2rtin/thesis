

\chapter*{Introduction}
\addcontentsline{toc}{chapter}{Introduction}

Providing information to consumers is a common task usually solved by implementing a web page or a mobile application.
Spoken dialogue systems on the other hand, are suitable for fulfilling such task while simplifying the access to the information.
Spoken dialogue systems allow people to interact with a computer the most natural way, by voice.
They are intuitive, direct and hands-free, which renders an opportunity for deployment in many fields.
There is no need to find a mobile device and enter queries into some puzzling interface, it is enough to simply say the words.
Despite this, it may serve as a valuable asset for obtaining information for the elderly and especially for blind people and people with visual impairment.%it addresses larger spectrum of people, convenient source of information

However, it is very difficult to implement a dialogue system from ground up.
Fortunately The Alex dialogue system platform\cite{asdf} (ASDF) incorporates all of the key components needed for performing the task.
We decided to take the advantage of ASDF and implement a spoken dialogue system in English for providing public transport information (PTI) in New York.
New York, the city with the highest ridership in the United States and with one of the most extensive subway systems in the world, presents a formidable challenge.
The English language benefits from wider range of use and larger competition to be compared with.
This enables us to build better systems with greater potential.

Automatic speech recognition (ASR) is not yet at the point where the meaning of every utterance could be reliably identified by a computer.\cite{asr}
Although with a restricted domain, it is possible to achieve substantially better results utilizing a trained speech recognizer.
A crowdsourcing platform CrowdFlower will be used to evaluate our PTI solution and to collect speech data.
The collected speech data will be facilitated to train a Kaldi speech recognizer and it will be compared with the Google ASR.
The comparison measures will be based on the Crowdflower contributor's report. % deliberation testimony

Our PTI solution will be a useful showcase of the ASDF and will further contribute to collect speech data for improving the quality of speech recognizers. \ask{should we mention MTA here?} %It will be competing in a MTA app quest competition.

The outcomes of this thesis include the public transport information telephone-based dialogue system for New York, \todo{and the other stuff}.

The paper is organized as follows.
In the first chapter we will look over the used technologies.
In the second chapter we will discuss the implementation in its principles by each component. %at a time
The third chapter will cover the features of implemented solution and display its capabilities.
The fourth chapter will describe the workflows associated with creating such dialogue system.
Training Language model and building Kaldi decoder will also be covered by this chapter.
In the fifth chapter we will draw conclusions and summarize the findings encountered in the process.




% telephone-based spoken dialog system

% The development of dialogue systems makes possible more natural human-computer interaction.
% The dialogue systems have prospects to be increasingly utilized at robotics and for providing information.
% Spoken dialogue systems are being used more frequently for their potential in numerous applications at many different fields.
% Spoken dialogue systems allow a user to interact with a computer using voice as the primary communication medium.

% They are being used for loan establishment, car rental, call rate adjustment etc.
% They may serve for the purposes of renting a car, establishing loans, or changing telephone rates, rezervace hotelu.


% Besides, there is a reason why in every other sci-fi movie, characters communicate with computers solely via voice. It is easy (simple?).



% -----------
% intro
% what's it about
% how to do the experiments
% dokumentace

% showcase systému, english - bigger potential -> better system
% dialog - všichni to dělaj na webovkách, telefonech, některý lidi to ani 

% nemůžou použít(slepý), proto je lepší dialog. 
% motivace - starý, slepí, pohodlnost, nechci vytahovat nic, jen se zeptám

% kapitola-proces, kdybych to dělal pro další doménu, tak co bych udělal 

% 1.2.3....
% tu finální, a potom v diskuzi - zkoušeli jsme to i takhle, ale 
% nefungovalo to pro to a tak.,

% process třeba jenom co obnáší natrénování modelu

% dialog-výhody, omezení
% obecně ds, alex,1:41 PM 4/10/2015 co všechno se musí adaptovat, jak se to musí udělat, co 
% jsem musel změnit u jednotlivých

% evaluace (cf), transkripce
% soutěž
% ----------




%Scientific paper consists of:

%	Introduction
%	Materials and Methods
%	Results
%	Discussion
%(co je známo, cíl práce, je to důležité, volba metody)
%at the end - a paragraph with a structure of the document
%THE INTRODUCTION HAS TO REFLECT THE PROCESS OF THE PAPER. WHAT EXACTLY WILL HE GET LATER SERVED AT THE TEXT.

%INTRODUCTION:

% What is a dialogue system
% Introduction to the problematics
% It is important
% method selection
% what has been done
% what has not been done
% what do we want to display in our work
% apt definition of the goals of this work

%This work is concerned with this and that, the assumptions are these, we would also want to use crowd sourcing for leaning something which enables this and that, then we evaluate


% What is interesting on this thesis: 
%  - It utilizes voice interface for obtaining information.
% Who could utilize output of this thesis:
%  - MTA for sure
% What output could be marked as surprising:
%  - we was able to achieve a comparable results with google TTS.
% What am i most proud of?
%  - The fact that we were able to participate in an MTA contest with a good competitive solution.
% If someone should continue my work - what parts should have he used and what should have he rewritten.
%  - He should use the whole solution, bus he would have to rewrite a SLU and HDC a bit for better handling of the streets and stuff
% What have I learned during the process, who could be interested:
%  - For sure the companies that make similar applications for different domains could be interested.
%  - The ones that want to use ADSF for development of their own dialogue system
%  - I have learned a great deal about dialogue systems, working with crowd sourcing etc.
