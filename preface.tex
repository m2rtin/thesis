\chapter*{Introduction}
\addcontentsline{toc}{chapter}{Introduction}

Providing information is a common necessity for every domain. Majority of většina řešení spočívá v napsání webových stránek nebo mobilní aplikace. Spoken dialogue systems můžou úspěšně zastát takovýhle task while obsloužej mnohem jednodušeji even larger spectrum of people. 
Dialogy jsou zajímavý a začínaj se objevovat na různých místech. 

v dialogových systémech je velký potenciál na poli closed domains na hromadě míst. Dnes se používají k taskům pro zřizování půjček, půjčování aut, nastavení tarifů. Vytvoření takového dialogu je náročný úkol a stejně tak náročný úkol je existující systém překlopit do jiné domény. Naším úkolem je překlopení do jiného jazyka v rámci stejné domény. To skýtá challenge neboť různé jazyky souvisí z různými národy a řečovými zvyky a kulturními odlišnostmi.

Spoken dialogue systems are being used more frequently for their potential in numerous applications at many different fields.

jak je to nejpřirozenější interface

konvenční přístup sdělování informací. Pro sdělování informací a jednoduchých tasků se běžně používají www stránky nebo mobilní aplikace. V dnešním světě kompjůtrů ale emerges další zdroj informací. dialogové systémy. Spoken dialogue systems can provide 

proč je to důležitý: protože to 

It allows visually impaired people to obtain information very easily, furthermore výhoda spočívá v jednoduchosti, easy as saying the words. 


\todo{že jsme si vybrali new york, protože má největší pti v americe a je tak dostatečně vhodný a challenge kandidát}

words 	
% Intro to problematics, sugar, why it is interesting.
Spoken dialogue systems become increasingly relevant in our everyday lives.
They are being used for accessing information at many different domains.
They may serve for the purposes of renting a car \cite{CarRental}, establishing loans \cite{Loan} or changing telephone rates \cite{Tarrif}, rezervace hotelu.
However, switching existing system form one domain to another is as difficult task as building brand new system from the beginning.
% Our task
Our task is to switch dialogue system from Czech to English language.
This brings challenge as different languages relate to different speech habits as much as different nationalities and cultures.
% Why is it important
English is important, there are much more data and potential customers in english.
It is relevant...Besides, there is a reason why in every other sci-fi movie, characters communicate with computers solely via voice. It is easy (simple?).


There are new dialogue systems for specific domains
Spoken dialogue system allow natural human interaction with a machine using voice. 

Spoken dialogue systems allow a human user to interact with a machine using voice as the primary communication medium.
A typical dialogue system consists of a speech understanding component, a dialogue manager, and a speech generation component.

Spoken dialogue systems are emerging as an intuitive interface for providing conversational access to online information sources

In this paper we present a dialogue system developed in python for a public transport information in new york. The system integrates spoken and written language processyng employing two components...Handwriten application forms filled out by propective customers for the issing of n insurance ocntract are processed blabla...The dialogue component, in turn, undertakes the collection of the missing information of the paplicatoin forms by the calls to the customers. 

-----------
intro
what's it about
how to do the experiments
dokumentace?

showcase systému, english - bigger potential -> better system
dialog - všichni to dělaj na webovkách, telefonech, některý lidi to ani 

nemůžou použít(slepý), proto je lepší dialog. 
motivace - starý, slepí, pohodlnost, nechci vytahovat nic, jen se zeptám

kapitola-proces, kdybych to dělal pro další doménu, tak co bych udělal 

1.2.3....
tu finální, a potom v diskuzi - zkoušeli jsme to i takhle, ale 

nefungovalo to pro to a tak.,

process třeba jenom co obnáší natrénování modelu

dialog-výhody, omezení
obecně ds, alex,1:41 PM 4/10/2015 co všechno se musí adaptovat, jak se to musí udělat, co 

jsem musel změnit u jednotlivých
evaluace (cf), transkripce
soutěž
----------



Human conversation is generally a natural, intuitive, robust and efficient means for interaction.
Spoken dialogue systems offers simple, direct, hands-free access to information.

\todo{tady je todo}


Scientific paper consists of:

\begin{itemize}
	\item Introduction
	\item Materials and Methods
	\item Results
	\item Discussion
\end{itemize}

INTRODUCTION:

\begin{itemize}
	\item What is a dialogue system
	\item Introduction to the problematics
	\item It is important
	\item method selection
	\item what has been done
	\item what has not been done
	\item what do we want to display in our work
	\item apt definition of the goals of this work
\end{itemize}


1 sentence general shieet, then right to it.
from general to concrete
(co je známo, cíl práce, je to důležité, volba metody)
at the end - a paragraph with a structure of the document


THE INTRODUCTION HAS TO REFLECT THE PROCESS OF THE PAPER. WHAT EXACTLY WILL HE GET LATER SERVED AT THE TEXT.
It is important, however there are these problem, thank god there is this...
This work is concerned with this and that, the assumptions are these, we would also want to use crowd sourcing for leaning something which enables this and that, then we evaluate

the output of this thesis is a dialogue system for new york.

\subsection{Elevators pitch} % (fold)
\label{sub:elevators_pitch}

What is interesting on this thesis: 
 - It utilizes voice interface for obtaining information.
Who could utilize output of this thesis:
 - MTA for sure
What output could be marked as surprising:
 - we was able to achieve a comparable results with google TTS.
What am i most proud of?
 - The fact that we were able to participate in an MTA contest with a good competitive solution.
If someone should continue my work - what parts should have he used and what should have he rewritten.
 - He should use the whole solution, bus he would have to rewrite a SLU and HDC a bit for better handling of the streets and stuff
What have I learned during the process, who could be interested:
 - For sure the companies that make similar applications for different domains could be interested.
 - I have learned a great deal about dialogue systems, working with crowd sourcing etc.


% subsection elevators_pitch (end)

 dialog system, which offers the promise
of simple, direct, hands-free access to information

 telephone-based spoken dialog systems

\subsection{The official task description} % (fold)
\label{sub:the_official_task_description}

% subsection the_official_task_description (end)

The goal of this thesis is to develop a dialogue system providing information about public transport in English. The system will be based on the Alex dialogue systems framework developed within the department. It will provide real-time transport information and will be evaluated with real users. For the purpose of evaluation, a crowd sourcing methods will be used, e.g. CrowdFlower.com. The evaluation measures will be based on subjective user satisfaction. 


Psutka, J. and Müller, L. and Matoušek, J. and Radová, V. : Mluvíme s počítačem česky. p. 752, Academia, Prague, 2006.
C.M. Bishop, Pattern Recognition and Machine Learning, vol. 4, no. 4. Springer, 2006, p. 738.

alt+f3 select all occurrences
ct