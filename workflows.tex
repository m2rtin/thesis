\chapter{Workflows - Development processes}

\section{Deployment}

\subsection{docker}

docker is a platform for rapid deployment of systems and it is an universal because it uses ubuntu inside. This allowed us to configure the image only once and than run it elsewhere. Development is also really usnadněný díky optionu -v, kterým jsme schopni propašovat libovolný adresář. So the workflow bypadal asi tak, že jsme měli i třeba starou verzi ptien zabejkovanou se všema dependencema a včkem jsme tam propašovali adresář s celým ADSF. na virtuálku jsme to dostali pomocí rsyncu, kterej updatoval pouze změněný adresáře. This allowed us really quick development loop, quick fixes of deployed sytem etc.

\subsection{metacentrum}

In order to provide service for more than one caller, we need to have multiple instances of our machine. We've used free cloud solution from cesnet metacloud, that provides services for students for free. This was extremely helpful, because the computation nároky jsou veliké.

pro každou instanci jsme použili nastavení se 4mi procesoty, 16ti GB paměti (stačilo by 8, ale stejný template jsme použili i pro trénování modelů, což vyžadovalo hodně paměti). Protože síť cesnet je spojená s matfyzáckou sítí, byla správa VMs velmi usnadněna.


\section{Creating CrowdFlower Job}

Assembling a Crowdflower job can be accomplished through one of many templates for ordinary tasks such as various data analysis,categorization, comparison, revision, review, transcription etc. %carried out realized, executed, leveraged
Custom and more sophisticated tasks can be carried out from scratch.
Crowdflower provides an interface for users to edit CrowdFlower Markup Language (CML), CSS and custom JavaScript that runs once on page load.
There is a possibility to inject custom HTML code as well.
The platform automatically inserts test question and evaluates contributors based on them.
CML and JavaScript are essential for utilizing Crowdflower's quality control

CML handles basic input controls that are necessary to utilize Crowdflower's quality control. % and crowdsourcing interfaces for named entity annotations 

It is best suited for the tasks that can not be automated and further split into smaller subtasks.
It is desirable that the tasks are as simple as possible to eliminate errors resulting from the lack of knowledge or misinterpretation.

However, repeated labeling is costly and may tend to be comparable in costs to the in-house solution.
CrowdFlower offers the option of screening users via quiz that takes place beforehand to determine quality of the worker.
%CrowdFlower and SamaSource  vs Crowdflower an Samasource


\section{Training Kaldi ASR}

pro natrénování kaldi je potřeba nainstalovat všechny devel prerequisities (SRILM) a pokud chceme ve finále použít ASDF toolkit pro evaluaci modelu, potřebujem i ALEX prerequisities. potom je potřeba vydumpovat databázi - z ní se vemou class labeled terminály, potom přidáme výstup z gramatiky a natrénujeme LM Potom potřebujeme Akustický model, v našem případě si ASDF stáhne aktuální ze serveru a natrénuje se kaldi model. potom se zvolí training/testing sety a otestuje se to v ASDF nebo něčím jiným.

\subsection{grammar}

pro dobrej rozpoznávací model je potřeba mít dobře rozložený slova, kterýma trénujeme LM, To se dá udělat tím, že se do toho nacpe spousta transkripcí a ono se to tam přirozeně přesype. My ale nemáme spoustu transkripcí. Proto jsme se rozhodli udělat bootsraping LM pomocí gramatiky.

Idea je taková, že vytvoříme gramatiku simulující věty, které by uživatel mohl běžně říct, věty, které očekáváme na vstupu. Tím pokryjeme ty nejčastější případy.

naše gramatika obsahuje jednoduchá pravidla O - option - speciální případ alternativy, A - alternativa - ,S (itemize), plus terminál.
Terminály jsou stringy, typicky hodiny, zastávky, města, období, 

takhle může vypadat například pravidlo A(S(Where, O(from), bla, bla)). 
Toto pravidlo se přepíše na .... 

Při přílišné snaze pokrýt všechny možností se snadno stane, že se vygenerují i věty, které nedávají smysl. Například: Chci jet v deset do prahy za půl hodiny s pěti přestupy. což je na škodu. Proto jsme se snažili generovat pouze validní vstupy, které dávají smysl.

Jako terminálý jsme použili databáze zastávek, které alex používá nativně.


\section{Process of changing domains} 
  -what would one change if he wanted to use this framework and use utilize it in different domain
  If I was to switch to a different domain, the following should have to take place.
  potřeboval bych si rozvrhnout co budu potřebovat v databázi, co budu používat v slučku a jak se mi bude moct měnit dialog, to znamená ideálně si nakreslit celej process dialogu na papír pomocí diagramů, z toho se dá vykoukat, co bude potřeba pro zodpovězení té které otázky. jaké api budu využívat pro přístup na internet, jaké keywords si budu potřebovat v paměti. potom vytvořit walking skeleton a nasadit. vyevalvovat....
  todo{má cenu tohle psát?} možná do lesson learned?
  
