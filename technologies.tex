\chapter{Technologies used}

\section{Alex spoken dialogue framework}

The Alex Dialogue System Framework (ADSF) serves for utilizing research in the development of spoken dialogue systems.
The ADSF is maintained by the dialogue systems group at UFAL \footnote{\url{https://ufal.mff.cuni.cz/grants/vystadial}}, the Institute of Formal and Applied Linguistics, Faculty of Mathematics and Physics, Charles University in Prague.
It is written in Python.

The ADSF consists of baseline components for assembling spoken dialogue systems.
There are tools for processing logs and evaluating spoken dialogue systems.
These tools can be used for audio transcriptions for example.
A small set of example implementations for different domains is also present.

There is a working Public Transport Information (PTI) \cite{ptics} for Prague public transport and the Czech Republic transport network in Czech language.
Our solution is based on the Czech version.
However, switching to English renders a challenge emerging from nation, culture and speech habit differences.
It also brings the advantage of having more versatile system deployable in different cities or countries even just by changing a knowledge base.
Collecting English data is also important for creating better models that can be facilitated by other applications within ASDF.

\todo{There are other services used for example for converting text to speech mentioned in the next chapter.}


\subsubsection{Automatic Speech Recognition}

Automatic Speech Recognition (ASR) transforms spoken words into text.
Many applications already use ASR technology as an interface between human and a computer, although it is not yet capable of understanding all speech in any environment.
Many factors influence perception of voice.

Acoustic conditions, voice differences, distance from the recording device, heavy accent, even voice emphasis, these are few of the issues versatile ASR has to cope with.
Very good overall performance delivers cloud-based speech recognition API by Google which can be used withing the ASDF.
Achieving better quality requires great many hours of transcribed text.

However, when we restrict the recognition to a specific domain, the scope of words becomes quite limited.
There is only so many expressions that can be used in a common conversation about particular subject.
With a recognizer trained on narrower domain, better results can be achieved.

%\subsubsection{Kaldi}

Kaldi is an open-source \footnote{Apache License 2.0} toolkit for speech recognition based on finite state transducers.
We use python wrapper Pykaldi within the ADSF for building a Kaldi decoder and effectively deploying trained ASR.
Kaldi decoder is requires statistical models, an acoustic model (AM) and a language model (LM).
The AM is trained within the department \cite{oplatek}.
It defines probabilities of acoustic features for a given word.
The LM is more domain specific as it refines probabilities of a word being recognized.
We use ASDF scripts that utilize the SRI Language Modeling Toolkit (SRILM) for training LMs.
The process of training LM and testing Kaldi will be expanded upon in chapter \ref{ch:workflow} on page \pageref{ch:workflow}.

\subsubsection{Voice Activity Detection}

In order to send a voice track to ASR processing, we need to be able to cut speech into logical units (sentences).
This role is performed by the Voice Activity Detection (VAD).
VAD separates the noise and silence from the speech.

\subsubsection{Text To Speech}

Text To Speech (TTS) makes an instantaneous impression as this is the first and in most cases the only output an end user is able to perceive. % comprehend
The ASDF supports multiple TTS alternatives, Google, Flite, SpeechTech and VoiceRSS.
We have utilized VoiceRSS\footnote{\url{http://www.voicerss.org/}}, the free online service.
The VoiceRSS API requires API key which is limited to per day requests.
With the ASDF caching most of the web requests, it should not be a problem.

\subsubsection{VoIP interface}

Our spoken dialogue system communicates with users over telephone.
The ASDF exploits a modified version of communicating library PJSIP\footnote{\url{https://github.com/UFAL-DSG/pjsip}} for implementing VoIP applications.
There is no need for registering a telephone number, for running a dialogue system it is suffice to enter SIP account details.
SIP account can be freely registered at numerous providers.

For accepting incoming calls from USA, a toll-free number was provided by the department.

\section{Crowdsourcing}

Crowdsourcing is a method for acquiring data by delegating work to a community of people.
In particular online communities tend to be employed for convenience.
By dividing tasks into smaller independent parts, one can eliminate the need for expert workers and therefore reduce costs associated with acquisition of the coveted data.
In some cases the cost savings can be a tenfold of what in-house solution may provide as mentioned in \cite{mturk}. % offer
However, this method can hardly achieve the quality or accuracy of expert workers.

Collecting speech data for training ASR models in English is easier with the help of crowdsourcing.
There are several crowdsourcing platforms connecting workers with work requesters such as
Amazon Mechanical Turk\footnote{\url{https://www.mturk.com/mturk/welcome}},
Samasource\footnote{\url{http://samasource.org/}},
CrowdFlower\footnote{\url{http://www.crowdflower.com/}}
and many more.
Samasource is a non-profit organization with a noble cause of lifting people out of poverty through digital work.
It does not, however, meet our need of employing native English speakers.
While Amazon Mechanical Turk would match our requirements, it is no longer available for non-US requesters.
With Crowdflower, we are able to implement a custom solution directly within the platform.

Crowdflower has mechanisms such as monitoring answer distributions and computing confidence score for maintaining quality of the output data. % that yield good quality assurance. the assurance of quality
They claim great amount of contributor force which promises prompt job resolution. % wide range 
The platform contains comprehensible templates for common tasks.
It features interface for building a custom job from scratch with a sensible support and demonstrative examples, too. % in form of examples, too.


% In short, the placement options means allowing placement at certain locations:

% h means here: Place the figure in the text where the figure environment is written, if there is enough room left on the page
% t means top: Place it at the top of a page.
% b means bottom: Place it at the bottom of a page.
% p means page: Place it on a page containing only floats, such as figures and tables.
% ! allows to ignore certain parameters of LaTeX for float placement, for example:

% \topfraction: maximal portion of a page (or column resp., here and below), which is allowed to be used by floats at its top, default 0.7
% \bottomfraction: maximal portion of a page, which is allowed to be used by floats at its bottom, default value 0.3
% \textfraction: minimal portion of a page, which would be used by body text, default value 0.2
% \floatpagefraction: minimal portion of a float page, which has to be filled by floats, default value 0.2. This avoids too much white space on float pages.
% topnumber: maximal number of floats allowed at the top of a page, default 2
% bottomnumber: maximal number of floats allowed at the bottom of a page, default 1
% totalnumber: maximal number of floats allowed at whole page, default 3