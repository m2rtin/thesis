\chapter{Technologies used}

In this chapter we will introduce main technologies that were involved in creating our dialogue system.

\section{Alex spoken dialogue framework}

\acf{ASDF}\footnote{\url{https://github.com/UFAL-DSG/alex}} serves for utilizing research in the development of spoken dialogue systems.
It is maintained by the dialogue systems group at \acf{UFAL} \footnote{\url{https://ufal.mff.cuni.cz/grants/vystadial}}, Faculty of Mathematics and Physics, Charles University in Prague.
And it is written in Python.

The \ac{ASDF} consists of baseline components for assembling spoken dialogue systems.
There are tools for processing logs and evaluating spoken dialogue systems.
These tools can be used for audio transcriptions or semantic annotation for example.
A small set of example dialogue system implementations for different domains is also present.

There is a working \acf{PTI} \cite{ptics} for Prague public transport and the Czech Republic transport network in Czech language.
Our solution is based on the Czech version.
However, switching to English renders a challenge emerging from nation, culture and speech habit differences.
It brings the advantage of having more versatile system deployable in different cities or countries just by changing a knowledge base.
Collecting English data is also important for creating better models that can be facilitated by other applications within the \ac{ASDF}.

%\todo{There are other services used for example for converting text to speech mentioned in the next chapter.}

\subsubsection{Automatic Speech Recognition}

\acf{ASR} transforms speech into text.
Many applications already use \ac{ASR} technology as an interface between human and a computer, although it is not yet capable of understanding all speech in any environment.
Many factors influence perception of voice.

Acoustic conditions, voice differences, distance from the recording device, heavy accent, even voice emphasis, these are few of the issues versatile \ac{ASR} has to cope with.
Very good overall performance delivers cloud-based speech recognition \acf{API} by Google which can be utilized withing the \ac{ASDF}.
Achieving better quality requires great many hours of transcribed text.

However, when we restrict the recognition scope to a specific domain, the amount of words the recognizer needs to handle becomes quite limited.
There is only so many expressions that can be used in a common conversation about particular subject.
With a recognizer trained on narrower domain, better results can be achieved.

%\subsubsection{Kaldi}

Kaldi is an open-source \footnote{Apache License 2.0} toolkit for speech recognition based on finite state transducers.
We use python wrapper Pykaldi\footnote{\url{https://github.com/UFAL-DSG/pykaldi}} within the ADSF for building a Kaldi decoder and effectively deploying trained \ac{ASR} \cite{oplatek}.
Kaldi decoder requires statistical models -- an \acf{AM} and a \acf{LM}.
The \ac{AM} is trained within the department \cite{oplatek_thesis}.
It defines probabilities of acoustic features for a given word.
The \ac{LM} is more domain specific as it refines probabilities of a word being recognized.
We use \ac{ASDF} 
scripts that utilize the \acf{SRILM}\footnote{\url{http://www.speech.sri.com/projects/srilm/}} for training \acp{LM}.
The process of training \ac{AM} and testing Kaldi will be expanded upon in Chapter \ref{ch:workflow} on page \pageref{ch:workflow}.

\subsubsection{Voice Activity Detection}

We need to be able to determine the end of the utterance for the computer to take turn and respond.
This role is performed by the \acf{VAD}.
\ac{VAD} cuts speech into sentences and sends it to \ac{ASR} for processing.
It separates noise and silence from the speech.

\subsubsection{Text To Speech}

\acf{TTS} makes an instantaneous impression as this is the first and in most cases the only output end user is able to perceive. % comprehend
The \ac{ASDF} 
supports multiple \ac{TTS} alternatives, Google, Flite, SpeechTech and VoiceRSS.
We have utilized VoiceRSS\footnote{\url{http://www.voicerss.org/}}, the free online service.
The VoiceRSS \ac{API} requires \ac{API} key which is limited to per day requests.
With the \ac{ASDF} 
caching most of the web requests it suffices our intents.

\subsubsection{VoIP interface}

Our spoken dialogue system communicates with users over a phone.
The \ac{ASDF} 
exploits a modified version of communicating library PJSIP\footnote{\url{https://github.com/UFAL-DSG/pjsip}} for implementing VoIP applications.
There is no need for registering a telephone number, for running a dialogue system it is suffice to enter \ac{SIP} account details.
\ac{SIP} account can be freely registered at numerous providers.

For accepting incoming calls from USA, a toll-free number was provided by the department.

\section{Crowdsourcing}

Crowdsourcing is a method for acquiring data by delegating work to a community of people.
In particular online communities tend to be employed for their convenience.
By dividing tasks into smaller independent parts, one can eliminate the need for expert workers and therefore reduce costs associated with acquisition of the coveted data.
In some cases the cost savings can be a tenfold of what in-house solution may provide as mentioned in \cite{mturk}. % offer
However, this method can barely achieve the quality or accuracy of expert workers.

Collecting speech data for training \ac{ASR} models in English is easier with the help of crowdsourcing.
There are several crowdsourcing platforms connecting workers with work requesters such as
Amazon Mechanical Turk\footnote{\url{https://www.mturk.com/mturk/welcome}},
Samasource\footnote{\url{http://samasource.org/}},
CrowdFlower\footnote{\url{http://www.crowdflower.com/}}
and many more.

Samasource is a non-profit organization with a noble cause of lifting people out of poverty through digital work.
It does not, however, meet our need of employing native English speakers.
While Amazon Mechanical Turk would match our requirements, it is no longer available for non-US requesters.
With CrowdFlower, we are able to implement a custom solution directly within the platform.

CrowdFlower has mechanisms such as monitoring answer distributions and computing confidence score for maintaining quality of the output data. % that yield good quality assurance. the assurance of quality
They claim great amount of contributor force which promises prompt job resolution. % wide range 
The platform contains comprehensible templates for common tasks.
It features a web interface for building a custom job from scratch with a sensible support and demonstrative examples, too. % in form of examples, too.

%It is best suited for the tasks that can not be automated and further split into smaller subtasks.

\section{Deployment}

Running a real-time dialogue system can claim considerable amount of system resources.
All of the components of the dialogue system in the \ac{ASDF} 
are separate processes.
Also the static knowledge-base may outgrow ordinary computer's memory.
Hence, it is only right to employ multi-processor machine with sufficient amount of memory.
%velký dostatek paměti je potřeba i pro trénování

\subsection{Docker}

Docker\footnote{\url{https://www.docker.com/}} is a platform for rapid deployment of applications.
It contains a packaging tool and a lightweight runtime.
An application wrapped in docker is easily portable to any laptop or \acf{VM}.
The chain of events leading from discovering a flaw and changing source code, to deploying compiled dialogue system can be excessively accelerated with docker.

Dockerfile is a specification file used for automating docker image builds.
It allows to specify a sequence of instructions executed on a base image in order to create a new one.
Built docker image with the dialogue system can be executed in an isolated container.

The \acf{ASDF} is now using docker which is very useful for resolving dependencies.
Our Dockerfile is based on the base \ac{ASDF} 
Dockerfile with instructions for downloading newest models and knowledge-base.
There is a \texttt{-i} flag for mounting any directory into a docker container.

\subsection{MetaCentrum}

In order to provide service for more than one caller at a time, we need to have multiple instances of our dialogue system.
Computing and storage resources of MetaCentrum\footnote{\url{http://www.metacentrum.cz/cs/}} can be used freely by all students of academic institutions in the Czech Republic.
Various \acp{VM} were deployed on MetaCentrum for our dialogue system groundwork.
%The possibility to use extensible resources was really helpful.
% It runs on OpenNebula platform

For each instance of MetaCentrum \ac{VM}, configuration with 4 processors and 16GB was used.
Half of the memory would be sufficient, however, the same \acp{VM} were employed for training language models which involves excessive memory usage.


% In short, the placement options means allowing placement at certain locations:

% h means here: Place the figure in the text where the figure environment is written, if there is enough room left on the page
% t means top: Place it at the top of a page.
% b means bottom: Place it at the bottom of a page.
% p means page: Place it on a page containing only floats, such as figures and tables.
% ! allows to ignore certain parameters of LaTeX for float placement, for example:

% \topfraction: maximal portion of a page (or column resp., here and below), which is allowed to be used by floats at its top, default 0.7
% \bottomfraction: maximal portion of a page, which is allowed to be used by floats at its bottom, default value 0.3
% \textfraction: minimal portion of a page, which would be used by body text, default value 0.2
% \floatpagefraction: minimal portion of a float page, which has to be filled by floats, default value 0.2. This avoids too much white space on float pages.
% topnumber: maximal number of floats allowed at the top of a page, default 2
% bottomnumber: maximal number of floats allowed at the bottom of a page, default 1
% totalnumber: maximal number of floats allowed at whole page, default 3