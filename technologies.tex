\chapter{Technologies used}

\section{Alex framework - PTICS}

The Alex Dialogue System Framework (ADSF) is used for utilizing research in the development of spoken dialogue systems.
ADSF is maintained by the dialogue systems group at UFAL \cite{ufal}, the Institute of Formal and Applied Linguistics, Faculty of Mathematics and Physics, Charles University in Prague.
It is written in Python.

The ADSF consists of baseline components for assembling spoken dialogue systems.
There are tools for processing logs and evaluating spoken dialogue systems.
These tools can be used for audio transcriptions for example.
A small set of example implementations for different domains is also present.

There is a working Public Transport Information (PTI) \cite{ptics} in Czech language.
Our solution is based on the Czech version.
However the switching to English renders a challenge emerging from nationality, culture and habit differences.
It also brings the advantage of com advantages 

angličtina je dobrá v tom, že se můžem měřit s ostatníma, můžem sbírat anglická data pro vytváření lepších anglických modelů za pomoci široké nabídky crowdsousingu. 


\subsection{Automatic Speech Recognition}

Automatic Speech Recognition (ASR) transforms spoken words into text.
Many applications already use ASR technology as an interface between computer and a person, although it is not yet capable of understanding all speech in any environment.
Many factors influence perception of voice.
Acoustic conditions, voice differences, distance from the recording device, heavy accent even voice emphasis, these are few of the issues versatile ASR has to cope with.
Achieving better quality requires large number of hours of transcribed text.
However, when we restrict ourselves to a specific domain, the scope of words becomes quite limited.
There is only so many expressions that can be used in a common conversation about particular subject.
Therefore better results can be achieved.

Google -> Kaldi

\todo{HMM(AM)->ASR<-N-grams(LM)}

\subsubsection{Kaldi}

Kaldi is an open-source \footnote{Apache License 2.0} toolkit for speech recognition based on finite state transducers.
We use python wrapper Pykaldi within the ADSF.

\todo{Process of training and testing Kaldi models will be explained in chapter [whoKnows].}

\subsubsection{Google}

Google has its own cloud solution of ASR which is used by Google command and Google Translate.

\todo{describe it a little bit - take inspiration from code}

\subsubsection{CloudASR}

\todo{Should I even say anything about this?}

\subsubsection{Voice Activity Detection}

In order to send a voice track to ASR processing, we need to be able to cut speech into logical units (sentences).
This role is performed by the Voice Activity Detection (VAD).

\subsection{Text To Speech}

Small but nonetheless crucial part of Dialogue System.
Makes an instantaneous impression as this is the first and in most cases the only output an end user is able to perceive. % comprehend
There are also variety of TTS providers. ADSF supports Flite, VoiceRSS and Google.

\subsection{Dialogue Manager}

\todo{it has to keep inner states for knowing the context, it is the soul of dialogue system}

\subsection{Natural Language Generation}


\subsection{VoIP interface}

\todo{SIP}


All of these components are connected together via hub in a star-like shape shown in figure \ref{fig:hub}.

\begin{figure}[ht]
	\centering
	\includegraphics[width=0.5\textwidth]{../img/todo.eps}
	\caption{A typical star-like shape configuration of dialogue system components}
	\label{fig:hub}
\end{figure}

\todo{odstavec a obrázek o tom, jak se typicky daj použít dohromady - na příkladu ptics - nebo něco lehčího?}



\section{Crowdsourcing}

Crowdsourcing is a method for acquiring data by delegating work to a community of people.
In particular online communities tend to be employed for convenience.
By dividing tasks into smaller independent parts, one can eliminate the need for expert workers and therefore reduce costs associated with acquisition of the coveted data.
In some cases the cost savings can be a tenfold of what in-house solution may provide \cite{Quality Management on Amazon Mechanical Turk} % offer
However, this method can hardly achieve the quality or accuracy of expert workers. 

There are several crowdsourcing platforms connecting workers with work requesters such as Amazon Mechanical Turk \footnote{\url{https://www.mturk.com/mturk/welcome}}, Samasource \footnote{\url{http://samasource.org/}}, CrowdFlower and many more.
Samasource is a non-profit organization with a noble cause of lifting people out of poverty through digital work.
It does not, however, meet our need of employing native English speakers.
While Amazon Mechanical Turk would meet our requirements, it is no longer available for non-US requesters.
With Crowdflower, we are able to implement a custom solution directly within the platform.

Crowdflower has mechanisms such as monitoring answer distributions and computing confidence score for maintaining quality of the output data. % that yield good quality assurance. the assurance of quality
They claim great amount of contributor force which promises prompt job resolution. % wide range 
The platform contains comprehensible templates for common tasks.
It features interface for building a custom job from scratch with a sensible support and demonstrative examples, too % in form of examples, too.



% In short, the placement options means allowing placement at certain locations:

% h means here: Place the figure in the text where the figure environment is written, if there is enough room left on the page
% t means top: Place it at the top of a page.
% b means bottom: Place it at the bottom of a page.
% p means page: Place it on a page containing only floats, such as figures and tables.
% ! allows to ignore certain parameters of LaTeX for float placement, for example:

% \topfraction: maximal portion of a page (or column resp., here and below), which is allowed to be used by floats at its top, default 0.7
% \bottomfraction: maximal portion of a page, which is allowed to be used by floats at its bottom, default value 0.3
% \textfraction: minimal portion of a page, which would be used by body text, default value 0.2
% \floatpagefraction: minimal portion of a float page, which has to be filled by floats, default value 0.2. This avoids too much white space on float pages.
% topnumber: maximal number of floats allowed at the top of a page, default 2
% bottomnumber: maximal number of floats allowed at the bottom of a page, default 1
% totalnumber: maximal number of floats allowed at whole page, default 3