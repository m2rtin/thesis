\chapter*{Conclusion}
\addcontentsline{toc}{chapter}{Conclusion}

This work presented the dialogue system providing public transport information for New York developed in the Alex dialogue system framework.
This involved creating a handcrafted spoken language understanding, dialogue manager and natural language generator components for adopting the public transport domain.
We collected static easy to update knowledge-base from various public transport providers in New York.
Additionally the dialogue system supports weather queries 

Crowdflower crowdsourcing platform was utilized for collecting subjective user satisfaction.
The dialogue system proved to be stable and beneficial in helping everyday commuters.
Furthermore it was capable of competing alongside commercial applications in MTA App Quest.

Crowdflower was also used for transcribing collected calls.
Language model was trained with the combination of transcribed calls and generated data with grammar simulating user input.
Consequently a Kaldi decoder was trained and compared with Google ASR.
It was proven that the Kaldi ASR achieves considerably better results than the Google ASR on the public transport domain.

\todo{expand}

%\todo{we have produced a working showcase of PTIEN capable of competing at new york MTA contest}
