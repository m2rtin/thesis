\chapter*{Conclusion}
\addcontentsline{toc}{chapter}{Conclusion}

This thesis presented a dialogue system providing \acf{PTINY} developed using the \acf{ASDF}.
This involved creating a custom handcrafted \acf{SLU}, dialogue manager and natural language generator for the public transport domain.
Bootstrapping sentences were used for creating the first operational system.
All of the involved components were further enhanced incrementally while the system was evaluated by real users.
%Further development of all involved components was incrementally enhanced while the system was evaluated by real users.
We collected a static, easy-to-update knowledge-base from the public transit providers in New York.
Additionally, the dialogue system supports weather and current time queries for the entire United States.

CrowdFlower crowdsourcing platform was utilized for collecting audio data.
The collected data were later transcribed using CrowdFlower platform as well.
A grammar capable of generating sentences likely to be used by the users for public transport information inquiries was created.
The purpose of the grammar was to substitute the lack of data needed for creating a good \acf{LM}.
With the combination of CrowdFlower and grammar data, a \ac{LM} was trained and subsequently Kaldi decoder was built.

The Kaldi \ac{ASR} was compared with the cloud-based Google \ac{ASR}.
It was shown that in a limited domain Kaldi is able to achieve notably better results than Google \ac{ASR}.
Aside from the comparison of both \acp{ASR} within \ac{ASDF}, feedback forms from CrowdFlower served as a subjective user satisfaction measure for the comparison of the dialogue system as a whole with different \ac{ASR} components.
It was shown that \ac{PTINY} achieves better results with Kaldi \ac{ASR}.
Moreover, the dialogue system proved to be stable and beneficial in helping everyday commuters.

The goals of this thesis were successfully completed and the solution was integrated with the \ac{ASDF}.
We proved that the \ac{ASDF} is suitable for creating spoken dialogue systems.
Furthermore the \ac{PTINY} was capable of competing alongside commercial applications in the 2014 \ac{MTA} App Quest 3.0.


\section{Acknowledgements}

Access to computing and storage facilities owned by parties and projects contributing to the National Grid Infrastructure MetaCentrum, provided under the programme ``Projects of Large Infrastructure for Research, Development, and Innovations'' (LM2010005), is greatly appreciated. We also thank Fred Concklin for assisting with the \ac{MTA} contest and to Ondřej Dušek, Ondřej Klejch, Ondřej Plátek and Lukáš Žilka for their useful comments and discussions.

%\todo{we have produced a working showcase of PTIEN capable of competing at new york MTA contest}
