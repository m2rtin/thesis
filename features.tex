\chapter{Features -- Public Transport Information for New York}

In this chapter we will describe the functionality of \todo{PTINY} as a whole.
%In this chapter we will describe what the \todo{PTINY} is able to process.
The features are derived from the DM capabilities which is the brain of the dialogue system, but each component has to oblige.
When describing how the dialogue system responds to a particular request, the SLU has to extract semantics from the text input, DM has to decide what to do and NLG has to generate text from the dialogue acts.

\section{Providing current time} \label{sec:time}

The user can make better route selection decisions based on the knowledge of current time.
This is why it is important for the system to be able to provide it.
As opposed to the Czech Republic, there are places at different time zones in the United States.
Therefore we decided to support current time queries specifying a city or a state for providing more accurate, localized time.
The state is enough information to receive a location specific time, however specifying a city is more accurate as some states occupy two timezones.
If only a city is specified, the DM will respond with a dialogue act requesting the state name, unless the city is not ambiguous.
In which case it will infer the state from the ontology.
An example of current time inquiry is shown in table \ref{table:time}.

\begin{table}[h]
\centering
\begin{tabular}{ | r | p{0.85\linewidth} | } \hline
	\textbf{Speaker} & \textbf{Utterance} \\ \hline
	\texttt{User} & \textit{``What is the time now in California?''} \\ \hline
	\texttt{System} & \textit{``In California, it is 7:01 AM, Pacific Daylight Time.''} \\ \hline
\end{tabular}
\caption[Current time in California]{Example dialogue for obtaining current time in California.}
\label{table:time}
\end{table}

The time zone data are received from the Google Time Zone API\footnote{\url{https://developers.google.com/maps/documentation/timezone/}}.
If the API is inaccessible for some reason it returns apology act with default computer time set for New York, which is in Eastern Time Zone.
Time zone names could be included in the ontology as additional data, however this way, there is no need to keep track of the daylight saving offsets instituted in some sates.

\section{Weather forecast}

It is comforting for the users to be able to obtain weather information at once.
Following the example of utilizing weather forecast in the Czech public transport information, we have implemented an English version.
OpenWeatherMap API\footnote{\url{http://openweathermap.org/api}} is used for receiving weather data.
%The DM takes the action based on gathered location data much like at providing current time \ref{subsec:time}.

Specifying the city is enough to receive a weather information with the state being inferred from the ontology much like when providing current time, described at section \ref{sec:time}.
Except if the city is ambiguous, then it is necessary to specify the state as demonstrated in table \ref{table:forecast}.
In addition to city and state, the weather inquiry can be defined by time either relative or absolute.

\begin{table}[h]
\centering
\begin{tabular}{ | r | p{0.85\linewidth} | } \hline
	\textbf{Speaker} & \textbf{Utterance} \\ \hline
	\texttt{User} & \textit{``What will be the weather like in five hours?''} \\ \hline
	\texttt{System} & \textit{``In five hours It will be forty six degrees and clear sky.''} \\ \hline
\end{tabular}
\caption[Weather inquiry by relative time]{Example dialogue for obtaining weather information by relative time specification.}
\label{table:weather}
\end{table}

The API allows to ask for the weather at a specific time as well as for the weather forecast for the whole day.
In the first example \ref{table:weather}, there is a weather inquiry specified only by time.
Default city of New York is used as a reference point.

\begin{table}[h]
\centering
\begin{tabular}{ | r | p{0.85\linewidth} | } \hline
	\textbf{Speaker} & \textbf{Utterance} \\ \hline
	\texttt{User} & \textit{``Tell me the tomorrow weather forecast for Denver.''} \\ \hline
	\texttt{System} & \textit{``Alright, in Denver tomorrow, please specify the state.''} \\ \hline
	\texttt{User} & \textit{``Colorado.''} \\ \hline
	\texttt{System} & \textit{``In Colorado, tomorrow it will be light rain, temperatures forty five to sixty one degrees.''} \\ \hline
\end{tabular}
\caption[Weather forecast for the next day in Denver, Colorado]{Example dialogue for obtaining weather forecast for the next day in Denver, Colorado.}
\label{table:forecast}
\end{table}

The second example \ref{table:forecast} shows inquiry of the weather forecast for the next day in Denver, Colorado.
The temperature units are in degrees Fahrenheit.

\section{Finding a connection}

%finding trip, transit, route, schedule
The prime asset of PTINY is the ability to effectively respond to transport connection requests.

\subsubsection{Connection restrictions}

The key restrictions are the location from and where to travel, which is either a city, borough, stop or an intersection of two streets.
The ambiguity of a waypoint is resolved in similar manner as in the case of time zone or weather queries.
%The dialogue system will ask to specify the waypoint by more general or specific area.
The DM tries to infer the waypoint and returns a request for either more general or more limited area specification if an ambiguity is found.

User can further specify the following criteria.

\begin{table}[h]
\centering
\begin{tabular}{ r | p{0.7\linewidth} }
	Criteria & Restrictive utterance example \\ \hline
	departure time & \textit{``I want to depart in ten minutes.''} \\
	arrival time & \textit{``I want to be there at five o'clock.''} \\
	preferred vehicle & \textit{``I want to go by bus.''} \\
	transfer count & \textit{``I want to transfer two times at most.''}
\end{tabular}
\caption[Restrictive criteria specification]{Connection search criteria paired with example utterances.}
\label{table:attributes}
\end{table}

Both departure and arrival times can be specified in absolute or relative form.
The nature of the DM allows to say these restrictions at once or one by one which makes for better utilization of the dialogue system.
If any of the key restrictions are missing, the system demands appropriate additional information.

After the system responds with a route proposition, the user can either further specify his query or ask about the following connection attributes.

\begin{table}[h]
\centering
\begin{tabular}{ r | p{0.7\linewidth} }
	Attribute & Inquiry example \\ \hline
	\textbf{origin} & \textit{``Where does it leave from?''} \\
	\textbf{destination} & \textit{``What is the destination?''} \\
	\textbf{arrival time} & \textit{``How long till i be there?''} \\
	\textbf{departure time} & \textit{``When does it leave?''} \\
	\textbf{transfer count} & \textit{``How many transfers are there?''} \\
	\textbf{transfer duration} & \textit{``How long do the transfers take?''} \\
	\textbf{trip duration} & \textit{``How long will i travel?''} \\
	\textbf{trip distance} & \textit{``What is the distance of the trip?''}
\end{tabular}
\caption[Details about provided connection]{Inquiry examples regarding different attributes of a given connection.}
\label{table:attributes}
\end{table}

The trips in New York can be very long, hence the sequence of instructions is exhaustive.
Thus the length of the trip informs not only about the mileage, but also about the number of stops to pass through before each transfer.
Whereas for the duration, system responses only with the time value as demonstrated in table \ref{table:distance}.
%Thus we have included the option to ask about the length of the trip, which tells not only the mileage, but also the number of stops to pass through before each transfer.

\begin{table}[h]
\centering
\begin{tabular}{ | r | p{0.85\linewidth} | } \hline
	\textbf{Speaker} & \textbf{Utterance} \\ \hline
	\texttt{User} & \textit{``I want to go from Central Park to Broadway, Queens.''} \\ \hline
	\texttt{System} & \textit{``From central park to Broadway in Queens, as a first possibility
	I have found the following connection.
	Walk to 5th Av and East 84th St. 
	Take the bus M4 at 1:55 PM from 5th Av and East 84th St towards Midtown Penn Station via Broadway via Fifth avenue.
	Get off at Fifth Av and West 60 - 59th St. Walk to 5th Av and 59th St.
	Take the subway N at 2:11 PM from 5th Ave and 59th St towards Astoria, Ditmars Blvd. 
	Get off at Broadway.''} \\ \hline
	\texttt{User} & \textit{``Tell me the distance of the trip.''} \\ \hline
	\texttt{System} & \textit{``The trip is 5.2 miles long, You will travel 10 stops from 5th Ave and East 84th St by bus M4, then it is 5 stops by subway N from 5th Av and 59th St.''} \\ \hline
	\texttt{User} & \textit{``Now, tell me the duration of the trip.''} \\ \hline
	\texttt{System} & \textit{``It takes about 26 minutes.''} \\ \hline
\end{tabular}
\caption[Particular connection with distance and duration inquiries]{Providing a connection from Central Park to Broadway, Queens is portrayed. Responses to distance and duration inquiries are demonstrated.}
\label{table:distance}
\end{table}

\subsubsection{The Google Directions API}

We use the Google Directions API\footnote{\url{https://developers.google.com/maps/documentation/directions/}} to acquire connection data.
For simple from-to queries we use free API accessible through HTTP and we only use API key for more restrictive queries with preferred vehicle and transfer count limitations.
This maximizes the utilization of the key before reaching a monthly fee threshold.
The transfer count is not directly mapped to an API request, the offered links are rather filtered by the criteria from the API response.
When no connection suites the restrictions, an apology utterance is issued.

\subsubsection{Choosing the alternative}

Users can browse through offered connections back and forth by requesting next, previous or by saying the explicit number of the connection if the current route does not satisfy their needs.
By default, there are four alternative connections at disposal, sorted by departure time.


\subsubsection{Alternatives in combination with criteria inquiries}

As some route descriptions may be quite long, it is handy to take just a peek at other connection attributes without the need to browse through them.
An example of peeking is demonstrated in table \ref{table:peek}
This can be used prior to selection of particular alternative connection.

\begin{table}[h]
\centering
\begin{tabular}{ | r | p{0.85\linewidth} | } \hline
	Speaker & Utterance \\ \hline
	\texttt{User} & \textit{``How long does the next link take?''} \\ \hline
	\texttt{System} & \textit{``The second connection takes about 51 minutes.''} \\ \hline
	\texttt{User} & \textit{``How long till the third connection arrives?''} \\ \hline
	\texttt{System} & \textit{``The third subway reaches Broadway in 1 hour and 17 minutes.''} \\ \hline
\end{tabular}
\caption[Alternative attribute peek]{Example of obtaining attributes from alternative connections.}
\label{table:peek}
\end{table}


\section{General cases}

The following examples apply to general dialogue.
Supporting features are presented.

\subsubsection{Conflict apology}

If, according to the ontology, a specified city is not located at given state, an apology utterance is produced.
Similarly a street or stop to borough or city conflicts are checked.
An example is shown at table \ref{table:conflict}.
This type of behavior occurs in all other tasks, too.

\begin{table}[h]
\centering
\begin{tabular}{ | r | p{0.85\linewidth} | } \hline
	Speaker & Utterance \\ \hline
	\texttt{User} & \textit{``Tell me the weather forecast for Las Vegas in Iowa''} \\ \hline
	\texttt{System} & \textit{``I'm sorry, but the city Las Vegas is not located in the state of Iowa.''} \\ \hline
\end{tabular}
\caption[Conflict example of incompatible waypoints]{Example dialogue that produces an apology utterance resulting from incompatibility based on static knowledge from the ontology.}
\label{table:conflict}
\end{table}

This situation can be settled by two different ways.
The conflict can be resolved by negation of the wrong value as shown in \ref{table:negation}.
Or the whole dialogue can be restarted as portrayed at table \ref{table:reset}.

\subsubsection{Negation}

If the system miscomprehended user's intent or the user decided to change his mind, there needs to be a way to reverse the input.
User can explicitly negate the wrong value which causes a relevant slot in the system to be erased.
The system either issues an implicit confirmation of the same slot with a different value from history, or if there is not any, it asks about the desired value of the slot as shown in table \ref{table:negation}.

\begin{table}[h]
\centering
\begin{tabular}{ | r | p{0.85\linewidth} | } \hline
	Speaker & Utterance \\ \hline
	\texttt{User} & \textit{``I want to go from Forty Sixth Street.''} \\ \hline
	\texttt{System} & \textit{``Alright, from Forty Sixth Street, Where are You heading?''} \\ \hline	
	\texttt{User} & \textit{``You know what, no, not from Forty Sixth Street.''} \\ \hline
	\texttt{System} & \textit{``Where are You leaving from?''} \\ \hline
	\texttt{User} & \textit{``Thirty Sixth Street.''} \\ \hline
	\texttt{System} & \textit{``You want to go from Thirty Sixth Street, where do You want to go to?''} \\ \hline
\end{tabular}
\caption[Changing input by negation]{An example of changing mind and subsequent correction of origin waypoint by negation.}
\label{table:negation}
\end{table}

\subsubsection{Context resolution}

The dialogue system needs to be able to interpret user's utterance in the context of previously spoken topic.
Partially this is achieved by keeping track of previous states, however, there are situations like the one portrayed in table \ref{table:cr}, where more sophisticated strategy needs to be engaged.

\begin{table}[h]
\centering
\begin{tabular}{ | r | p{0.85\linewidth} | } \hline
	Speaker & Utterance \\ \hline
	\texttt{System} & \textit{``Which stop do You want to depart?''} \\ \hline
	\texttt{User} & \textit{``Miami.}'' \\ \hline
	\texttt{System} & \textit{``Alright, from Miami, where do You want to go to?''} \\ \hline
\end{tabular}
\caption[Context resolution of origin waypoint]{Example of resolving city from the origin request.}
\label{table:cr}
\end{table}

In the example, the system asks about the initial stop.
The user replies with a city without a preposition.
Even though the system was asking about a stop, the DM needs to deduce that the user does not want to go from particular stop, but rather from the city of Miami.
These automatic relations are defined in the ontology.
%protože věděl na co se ptal a vyinferoval, že uživatel myslí z nbebo from nebo in. rather than just city


% \subsubsection{Keeping track of previous states}

% It is important to know what was the topic in the previous utterances to correctly answer inquiries.
% This is fundamental not only for responding to details about particular connection, but also for other tasks, too as shown in table \ref{table:history}.
% \ask{is this a good example, shouldn't this be deleted and mentioned somewhere w/o the example - the ``orthogonal queries'' is maybe the second side of the same coin}

% \begin{table}[h]
% \centering
% \begin{tabular}{ | r | p{0.85\linewidth} | } \hline
% 	Speaker & Utterance \\ \hline
% 	\texttt{User} & \textit{What is the weather like in Saint Louis, Missouri?} \\ \hline
% 	\texttt{System} & \textit{In Saint Louis, It is eighty degrees and clear sky.} \\ \hline
% 	\texttt{User} & \textit{What is the time there?} \\ \hline
% 	\texttt{System} & \textit{In Missouri, it is 2:32 PM, Central Daylight Time.} \\ \hline
% \end{tabular}
% \caption[Keeping track of previous utterances]{Example of utilizing previous utterances for the purpose of providing a sensible response between two tasks.}
% \label{table:history}
% \end{table}

\subsubsection{Orthogonal queries}

It is essential for the system to be aware of the previous utterances to produce correct responses.
However, as the history is being kept, it may lead to misunderstandings.
An obvious confusion may occur when asking about orthogonal queries as showed in table \ref{table:confusion}

\begin{table}[h]
\centering
\begin{tabular}{ | r | p{0.85\linewidth} | } \hline
	Speaker & Utterance \\ \hline
	\texttt{User} & \textit{``What is the weather like in Saint Louis, Missouri?''} \\ \hline
	\texttt{System} & \textit{``In , It is eighty degrees and clear sky.''} \\ \hline
	\texttt{User} & \textit{``What is the weather like in Alaska?''} \\ \hline
	\texttt{System} & \textit{``I am sorry, I don't understand, the city Saint Louis is not located in Alaska.''} \\ \hline
\end{tabular}
\caption[Orthogonal time queries]{Example of location confusion from the previous utterance.}
\label{table:confusion}
\end{table}

From the example it is evident that the system did not forget the user mentioned the city of Saint Louis, which was considered relevant in the next turn, where it did not pass the compatibility test based on knowledge from the ontology.
Which of course leads to an apology utterance as mentioned earlier and can be resolved either by negation or by reset of the dialogue.

%e vidět, že se uživatel ptá na počasí v městě X, o kterém se vyinferuje, že se nalézá ve státě Y, když se pak ale zeptá uživatel na město Z, přepíše se v DM slot in\_city a systém se domnívá, že se jedná o město Z ve státě Y, načež zareaguje conflict omluvou.

\subsubsection{Selection and confirmation of slot values}

When the confusion network in DM contains more values of the same slot that have non-zero probabilities, it has to decide what value is intended by the user.
The system will either produce a selection or confirmation utterance based on the probability distribution of the values at particular slot.
The selection utterance is challenging the user to decide between two values with same probability.
Whereas the confirmation is a yes-no question about slot with a concrete value.

\subsubsection{Reset of the dialogue}

The dialogue system may get to a point, where its confusion network contains many uniformly distributed values within the same slot and it asks the user for resolving the correct value.
Whereas the user may consider the matter already closed.
This may for example arise when user, after receiving a connection link, wants a different one.
The system may no longer be able to appropriately respond to user's requests and altering values by negation is not effective.
When such situation occurs, the system can be restarted by saying a phrase \textit{``new entry''} or \textit{``restart''} or similar phrase indicating new entry.
Reseting erases all the slots in the dialogue system and it prompts the user to start asking from the beginning.

\newpage

\subsubsection{Supplementary intents} %auxiliary, additional, accompanying

The dialogue system has to be able to process speech habits commonly occurring in every dialogue.
In the table \ref{table:sup} there are enlisted all of the supplementary intents of the caller that PTINY is able to process.

\begin{table}[h]
\centering
%\small
%\hspace*{-1pt}\makebox[\linewidth][c]{
\begin{tabular}{ r | p{0.73\linewidth} }
	\textit{Intent} & \textit{Cause and action taken by PTINY} \\ \hline
	\texttt{greeting} & Courtesy act and prompt for inquiries.\\
	\texttt{farewell} & Indicates parting and an intent to hang up.\\
	\texttt{courtesy} & Usually after satisfactory response it concludes a task. The system will encourage user to ask further questions. \\
	\texttt{help requested} & Provides context sensitive help by randomly selecting a subtopic and saying how to specify a appropriate query. \\
	\texttt{not understood} & Indistinct ASR input results in an apology and a suggestion to repeat the last utterance.\\ %offer propose, urge, rocommend
	\texttt{silence} & Nothing has been said for a while. The system will ask if caller is still in the presence. % on the other side.
\end{tabular}
%}
\caption[Response actions for supplementary intents]{Response actions for supplementary intents}
\label{table:sup}
\end{table}

The help context is based on what the user was talking about in previous turns.
If for example the conversation was about finding a connection, it may suggest help in the form shown in table \ref{table:help}.

\begin{table}[h!]
\centering
\begin{tabular}{ | r | p{0.85\linewidth} | } \hline
	Speaker & Utterance \\ \hline
	\texttt{User} & \textit{``Help.''} \\ \hline
	\texttt{System} & \textit{``You can narrow your search by limiting the number of transfers.
	Just say, I want a direct connection, for example.''} \\ \hline
\end{tabular}
\caption[Context sensitive help]{Example of context sensitive help utterance.}
\label{table:help}
\end{table}


%máme dvě verze rozloučení, jednu pro sbírání dat z CF, která je vidět na \ref{table:tg1} a potom ordinary rozloučení, které místo vydání kodu se ptá na závěrečnou otázku, zda uživatel našel co potřeboval, tím můžeme potom z logů dostat user satisfaction based on Yes/no values.




% accessory
% whereas
% hence