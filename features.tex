\chapter{Features - Public Transport Information for New York}

in this chapter we will describe what the PTIEN is capable of. The features are derived from the DM capabilities, which is a brain of the dialogue system, but each component has to chip in. When we describe how the DS is capable of responding to a concrete request, the SLU has to extract semantics from the text input, DM has to what to do and NLG has to generate text from the dialogue acts.


\section{Features of PTIEN}

In this section we will describe the functionality of PTIEN as a whole.

\subsection{Accompanying sentences}

tabulka:
pozdravy, -  \\
ticho -  \\
rozloučení -  \\
nerozuměli jsme -  \\
help requested - context help provision \\
poděkování - reakce a zeptání se na další úkoly \\
restart - restart dialogu \\

to že jsme nerozuměli může být buť tím a nebo tím,
context help vychází z toho, o čem naposledy člověk hovořil, vybere se náhodně nějaká z vět o tématu, třeba že se můžou zepat na počasí v jejich rodném městě, nebo pokdud se bavili o PTI, tak : můžeš se třeba zeptat kolik to má přestupů...




\subsection{tasks}

\subsection{Providing current time} \label{subsec:time}

The user can make better route selection decisions based on what time it is.
This is why it is important for the system to be able to provide current time.
As opposed to the Czech Republic, there are places at different time zones in the United States.
Therefore we decided to support queries specifying a city or a state for providing more accurate, localized time.
The state is enough information to receive a location specific time, however specifying a city is more accurate as some states occupy two timezones.
If only a city is specified, the DM will respond with a dialogue act requesting the state name, unless the city is not ambiguous.
In that case it will infer the state from the ontology.

The time zone data are received from the Google Time Zone API\footnote{\url{https://developers.google.com/maps/documentation/timezone/}}. With inaccessible API it returns apology act and default computer time for New York, which is Eastern Time Zone.
Time zone names could be included in the ontology as additional data, however this way, there is no need to keep track of the daylight saving offsets used in some sates.

\subsubsection{Weather forecast}

Following the example of utilizing weather forecast in the Czech PTI, we have implemented an English version.
It is comforting for the users to be able to obtain weather information at once.
OpenWeatherMap API\footnote{\url{http://openweathermap.org/api}} is used for collecting weather data.
The DM takes the action based on gathered location data much like at providing current time \ref{subsec:time}.
In addition to city and state, the weather forecast can be specified by time.
The output dialogue act contains a range of temperatures as well as weather condition for given restrictions.

\subsubsection{Finding connection}

%finding trip, transit, route, schedule
The prime asset of our dialogue system is the ability to effectively respond to transport connection requests.
%\todo{nějakej žvást o vyhledávání spojů}
The key restrictions are the location from and where to travel, which is either a city, stop or an intersection of two streets.
The ambiguities of waypoints are resolved in similar manner as in time zone or weather forecast case.
The DM tries to infer city or borough and returns a request for waypoint specification only if an ambiguity is found.
If according to ontology, the specified street or stop is not located in a given borough or city, an apology dialogue act is produced.
User can further specify the departure or arrival time both in absolute and relative forms, a preferred vehicle and the maximum number of transfers.
The nature of the DM allows to specify these restrictions at once or one by one which makes for better utilization of the dialogue system.
If any of the key restrictions are missing, the DM issues a dialogue act demanding appropriate additional information.
After the DM responded with a route proposition, the user can either further specify his query or ask about the connection attributes.
This includes the origin, destination, arrival and departure times, number of transfers and the duration of the trip.
The trips in New York can be very long, hence the sequence of instructions is exhaustive.
This is why we have added the option to ask about the length of the trip, which tells not only the mileage, but also the number of stops to pass through before each transfer.
The user can also browse through offered connections back and forth by requesting next, previous or explicit number of the connection in case the current route does not satisfy user's needs.
For a given time, there are four alternative connections on offer by default.

We use the Google Directions API\footnote{\url{https://developers.google.com/maps/documentation/directions/}} to acquire connection data.
For simple from-to queries we use free API accessible through HTTP and we only use API key for more restrictive queries with preferred vehicle and transfer count limitations.
This maximizes the utilization of the key before it reaches a monthly fee threshold.
The transfer count is not directly mapped to an API request, it is rather computed from the API response.
When no connection suites the restrictions, an apology dialogue act is produced.
%advanced capabilities, when transfer count or preferred vehicle type is specified


%nasbíráme vstupy do confusion network a podle pravděpodobnostního rozdělení se rozhodne o faktech, z nich se pak začne 


\todo{context resolution, compatibility conflict example}

% \todo{after submitting let FJ lay critique on ya!}


% accessory
% whereas
% hence