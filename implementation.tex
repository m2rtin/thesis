\chapter{Implementation - Public Transport Information for New York}

In this chapter we will go through each component of our dialogue system we needed to arrange.
This does not include automatic speech recognition, text to speech nor VoIP interface.
%These components are ready to use.
The principle will be described along with relevant instruments. %apparatus. 
All of these components are domain specific descendants of domain independent components of the Alex spoken dialogue framework (ASDF).

First we will cover keyword database and matching words against the input, than we will describe changing states in our dialogue system.
Than we will explain how the state is translated to the output which will conclude in a summary of what the processing of any utterance looks like.

\section{Spoken Language Understanding}

To be able to process, evaluate and respond to user requests, we need to extract semantic meaning from utterances. %ASR hypothesis
This is realized via Spoken Language Understanding (SLU), which uses a vast static keyword database for analyzing words and phrases of each utterance.
Being able to handle such semantic representation makes it possible to change state of the dialogue system.
There are different approaches for SLU development.
There are SLU techniques based on statistical models learned from data.
We have implemented handcrafted SLU, based on simple keyword rules.
Both approaches are supported by the ADSF as demonstrated in \cite{slu}.

% Both approaches can are supported by the ASDF.
% \todo{which gives us direct control over the translation process.} % clear overview of the process

\subsubsection{Keyword database} 
% \todo{is this knowledge-base?}

Public transport information domain demands the ability to respond to two major constraint queries - location and time.
The time and supplementary keywords can be defined explicitly or generated by a simple script.
However, the location data are specific for the region we decided to cover and therefore must be gathered.

We ultimately need just the name of the stop for keyword matching. 
However, the stop or city names might be ambiguous which is why we need to keep further knowledge of geographic information and more general realm. %area territory locality region

%For a prompt keyword matching, there is an object kept in memory containing all of the location keywords as well as other PTI related keywords. class label database
\subsubsection{Location data terminals} \label{sec:terminals}

The types of location are streets, stops, boroughs, cities and states.
All of these location categories are listed in a separate file for the convenience of adding new or updating existing entries.
It is obvious that the borough list will be very narrow, may be even unnecessary because there is only five boroughs in New York.
But the idea is to be able to distinguish between streets and stops with the same name that are very likely to appear within the same city.
If we decided to expand the system to cover Los Angeles region, we might need to add not only LA boroughs but to specify a finer administrative division altogether.

As for the stops, we collected the latest data from
MTA\footnote{Metropolitan Transportation Authority - \url{http://www.mta.info/}},
PATH\footnote{Port Authority Trans Hudson - \url{http://www.panynj.gov/}},
NJ Transit\footnote{New Jersey Transit - \url{http://www.njtransit.com/}},
NY Waterway\footnote{New York Waterway - \url{http://www.nywaterway.com/}} and
Amtrak\footnote{The National Railroad Passenger Corporation - \url{http://www.amtrak.com/home}} for long-distance trips.
Most of the companies are providing their schedules for developers in a unified format.
The General Transit Feed Specification (GTFS) defines a common format for public transportation schedules and associated geographic information.\footnote{GTFS - \url{https://developers.google.com/transit/gtfs/}}

We could have adopted the GTFS format which would have been very convenient for updating.
However, some of the datasets were not strictly following the GTFS making it unfeasible to work with.
Missing values, overflowing columns or disunited expressions occurred infrequently, nevertheless throughout notable portion of the data.
The benefit of an easy update and access to additional information did not outweigh the shortcomings encountered.

We opted for a simple format that takes into account only the most important features.
Selecting fewer columns makes it easier to add places that are not available in the GTFS.
This includes some popular sites as well as few smaller transport companies commuting between tens of terminals.

In addition to official stops, we added over a hundred of the most popular sites in New York from various top n lists.
Those can be used as a good reference points in everyday commutes so we can for example handle sentences like:
\textit{``From Chrysler building to Columbus Park.''}.
We used Google Geolocation API\footnote{\url{https://developers.google.com/maps/documentation/business/geolocation/}}, to obtain longitude, latitude and borough for each popular site.

The obtained data in raw form can not be used for keyword matching. As opposed to the Czech language, there is not a necessity to take inflection into account, however there is a number of ways to express a stop or a street.
Stops in particular, mostly called after intersections, can be unfolded and expressed in different order and coupled with a different conjunction.
For example the stop \texttt{1 Av/E 111 St} can be expressed as \textit{``first avenue and east hundred and eleventh street''} as well as \textit{``east hundred eleventh street at first avenue''}.
Also the data contain numbers and unpronounceable characters like parenthesis, slashes, dashes and also abbreviations that are not unified.
For example the \texttt{St} can mean both \textit{street} and \textit{saint} and to continue, the word \textit{expressway} is abbreviated by \texttt{ep}, \texttt{ex}, \texttt{exp}, \texttt{expy} and \texttt{expwy}.
Thus for each category we have a separate file with possible forms generated by an expansion script.


\subsubsection{Dialogue Act Scheme}

Intents of the user as well as actions of the spoken dialogue system are represented by Dialog Acts (DA).
They consist of one or more Dialogue Act Items (DAI) that are elementary semantic information units.
DAIs are defined by a type, slot name and slot value.
The slot name and value are domain specific and further define the meaning.
In our case slot names may refer to a place for instance.
Exemplary Dialogue Act is shown at \ref{table:utterance}. We can see that the \textit{When does} and \textit{leave} correspond with the request DAI and that the inform is gathered from the word \textit{bus} in the sentence.

% Please add the following required packages to your document preamble:
% \usepackage{booktabs}
\begin{table}[h]
\centering
\begin{tabular}{ r | l }
	\textbf{Utterance} & \textit{When does the bus leave?} \\
	\textbf{Dialogue Act} & \texttt{request(departure\_time)\&inform(vehicle="bus")}
\end{tabular}
\caption{Example of semantic notation of an utterance}
\label{table:utterance}
\end{table}

Sometimes it is not clear how an utterance should be transformed into DA due to unknown context or ASR lapse.
The ADSF contains a Dialogue Act Confusion Network that deals with this issue. %helps with dealing
The confusion network stores a probability for each DAI and it presents the most likely DA based on the probability distribution of DAIs.
%The confusion network has a probability for each DAI and it predicts whether the DAI is present in the most likely DA or not.
%A DAI will be present in the result DA if the probability is greater than \( \frac{1}{2} \).

A confusion network is best utilized when processing ASR n-best hypothesis and using statistical SLU.

\subsubsection{Handcrafted SLU}

Handcrafted SLU works only with 1-best hypothesis from the ASR.
%We take into account only the first (best) hypothesis from ASR.
After an utterance is passed into our SLU, it is matched against class labeled database keywords and an abstract utterance marked with labels is produced.
Each label has a special parsing procedure that yields Dialogue Acts into the dialogue act confusion network.

%We We treat individually the following class labels:
A search for semantic meaning takes place in designated routines, the following class labels are handled separately.

\begin{itemize}
	\item \texttt{NUMBER} - Parsing hour and minute values and time fractions.
	\item \texttt{PLACE} - Parsing waypoints from stop, street, borough, city and state values.
	\item \texttt{TIME} - Absolute and relative time periods matching.
	\item \texttt{TASK} - Conversation topic  which is either weather or finding connection.
	\item \texttt{VEHICLE} - Preferred means of transport matching.
\end{itemize}

Due to the iconic Manhattan street grid, people in New York are likely to know their position based on the street and avenue names, which are commonly numbers.
They may not know the closest bus or subway station.
Therefore we decided to support streets as valid input for finding connections.
The idea is to let users specify an intersections rather than stops. Stops however, make more accurate search queries possible, because we have access to the latitude and longitude values mentioned earlier at \ref{sec:terminals}.
The ambiguity of streets and stops is not negligible, hence boroughs are also parsed as waypoint entries.

%po těchhle rutinách to jde ještě do dlouhý procedury, kde se matchujou klíčová slova a fráze v celé utteranci a další dialogitemy se přidaj do confusion network, to jsou věci jako pozdravy, courtesy, handlování takovejch speciálních promluv pro zopakování, courtesy, pozdravy, souhlasy, ale i doprovodné requesty a sdělení s počty přestupů, speficikací transport query,  if else se matchujou kombinace a posloupnosti slov
Further series of matching steps take place after those routines.
We search for keywords and phrases in the whole utterance regardless of the context and add more DAIs to the dialogue act confusion network based on simple if-else rules.
It handles particular utterances for courtesy, greeting, acknowledgement as well as requests and notifications about transport query restrictions.
Also DAIs from non-speech events like silence or noise DAIs are extracted.

\section{Dialogue Manager}

A Dialogue Manager (DM) is a component responsible for processing and changing dialogue states in order to take appropriate actions in response to the user's query.
The history of the dialogue and inner states are recorded for better comprehension of current query.

\subsubsection{Ontology}

Ontology contains a static domain knowledge information that can be used for better understanding relations between entities.
It defines dialogue act item slot types and values from database mentioned at \ref{sec:database} and relationships between them.
This allows DM to gain more relevant information for example by context resolution.

In addition, it provides relations between locations for discovering compatibility conflicts and for implicit value inference.
The compatibility lists are bidirectional and concern street-borough, stop-borough and city-state relations.
%Stop geolocation is stored in the ontology.

\subsubsection{Handcrafted DM}
%handluje tam ty pravděpodobnosti

Our handcrafted DM extracts facts from the combination of its inner states, history of the dialogue and the DAI probability distribution taken over from the dialogue act confusion network from SLU.
In an if-else rule block it selects a subroutine for deciding what will be the next action taken.

Simple responses to elementary facts are among the first served by the rule block.
Those include actions for greetings, repetition of the last system utterance, a context specific help, reseting the system or a back-off action.
In case the user's turn yields no change since the last time, or the input from ASR was invalid, the DM executes a back-off action, which is randomly selected from providing help, repeating the last utterance, silence and an act for saying it simply did not understand.
The reset action, for example, can be used when user asks about two unrelated topics, the DM mistakenly combines facts from both and 
produces unexpected responses.
\todo{leave this to the feature chapter and add an example for this behavior}

\section{Natural Language Generation}

Natural Language Generation (NLG) component transforms inner states of the dialogue system into readable text form.
Having a limited domain, we are able to cover NLG by a template dictionary that has entries for each dialogue act item and combinations of some dialogue act items.
Seamless communication can be achieved by constructing adequate NLG templates.
The slot value of each dialogue act item is treated as a variable that can be inserted in the translated sentence.
An example of NLG translation is displayed at \ref{table:nlg}.
It is evident how the slot value \textit{Broadway} is injected into the template and also that the time value is translated to word representation.

\begin{table}[h]
\centering
\small
\hspace*{-3pt}\makebox[\linewidth][c]{
\begin{tabular}{ r | l }
	\textbf{Dialogue Act} & \texttt{inform(to\_stop={"Broadway"})\&inform(arrival\_time={"04:26:PM"})} \\
	\textbf{NLG template} & \texttt{inform(to\_stop=\{to\_stop\})\&inform(arrival\_time=\{arrival\_time\}):} \\
	 & \hspace{0.5cm} \texttt{"It arrives at \{to\_stop\} at \{arrival\_time\}."} \\
	\textbf{NLG output} & \textit{It arrives at Broadway at four twenty six P M.}	
\end{tabular}
}
\caption{Translation example of dialogue act to sentence by Natural Language Generation component}
\label{table:nlg}
\end{table}

There may be multiple expressions defined for each dialogue act, which is useful for making overused dialogue acts, such as greetings, seem more natural and less robotic.
The NLG templates can be overlapping and proper translation rule has to be searched for.
The search proceeds from exact to general and from long to short sequences of dialogue act items.

\section{Main System Hub}

The central hub gives the dialogue system modularity.
All of the components are connected together via main hub in a star-like shape shown in figure \ref{fig:hub}.
Each component runs as a separate process and the hub essentially chains them via standard stream pipelines as shown in \ref{fig:chain} and coordinates their continuity.
% the hub handles incoming call events


\begin{figure}[ht]
\centering
\begin{subfigure}{0.5\textwidth}
  \centering
  \includegraphics[width=0.4\linewidth]{../img/star.eps}
  \caption{Hub configuration}
  \label{fig:hub}
\end{subfigure}%
\begin{subfigure}{0.5\textwidth}
  \centering
  \includegraphics[width=0.4\linewidth]{../img/hub.eps}
  \caption{Component chain}
  \label{fig:chain}
\end{subfigure}
\caption{On the left there is a typical star-like shape configuration of a dialogue system components and the right figure shows the inner component chain of the central hub.} % inner -> piped
\label{fig:test}
\end{figure}
