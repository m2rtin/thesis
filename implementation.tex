\chapter{Implementation - Switching to English - Converting}

\section{Database}

class label database
věci se držej v paměti, různý tvary tam i zpět

category labels - kategorie různejch věcí
ontologie

Database has to consist of terminal symbols that we can parse from the input. 

musí obsahovat všechny možný travry. Tady je v nevýhodě čeština, neboť ta musí obsloužit všechny pády a rody, naproti tomu anličtina, ta pády nemá, ukázalo se však, že terminály jsou sestaveny z mnoha částí, které odpovídají křižovatkám nebo 

\subsection{Terminals}

\todo{How is PTICS different from PTIEN}

\section{Spoken Language Understanding}

Spoken Language Understanding (SLU) extracts semantic meaning from speech utterances (ASR hypothesis) and translates them to an interpretable representation.
Being able to handle such semantic representation makes it possible to change state of the dialogue system.
There are different approaches for SLU development.
There are SLU techniques based on statistical models learned from data, however we have implemented handcrafted SLU based on simple keyword rules. \todo{which gives us direct control over the translation process.} % clear overview of the process

In the ADSF we are able to use both \todo{polish and cite boostraping slu?}, however we have used the handcrafted SLU only.
\todo{expand on the dialogue act items, dialogue confusion network etc? ve spojitosti s database}

We take into account only the first (best) hypothesis from ASR.
After an utterance is passed into our SLU, it is matched against class labeled database keywords and an abstract utterance with class labels are produced.
Each class defines a special parsing procedure that yields Dialogue Acts into the Confusion Network.
We recognize the following class labels:

pro následující klásy máme speciální rutiny

\begin{itemize}
	\item \texttt{NUMBER} - context determines further meaning
	\item \texttt{PLACE} - stop, street, beginorough, city, state slots
	\item \texttt{TIME} - absolute, relative, am, pm time slots
	\item \texttt{TASK} - weather, find connection slots
	\item \texttt{VEHICLE} - type of vehicle slot
\end{itemize}


\todo{kde mám zmínit kategorie - informy, iconfirmy, requesty a jestli vůbec?}

čísla - speciální handlování pro minuty, hodiny, části hodin, například půl pátý a pět minut 
place - speciální handlování spočívá v kontextu, jestli se tam vyskytujou další place - sloty, tak první je od, pak do
time - podle kontextu se určí, jestli chce odjíždět, nebo přijíždět, absolutní/relativní čas, negace
task - pouze negace - nechci počasí
na začátku se taky parsujou non-speach eventy - ticho, noise, other - nerozpoznáno

nakonec se projde utterance znova a vyzobou se kywordy pro ostatní akty, jednoduchá pravidla pro počet transferů atd.. for example a může bejt pravidlo v jako figure?

podle category labelů se provádí parsování různých druhů věcí, potom se taky hledají obecné keywordy pro naparsování aktů jako je:distance,...

Also from keywords keyword parsing 

meta keyword parsing - distance, duration, transfer count, courtesy keyword lookup

\section{Policy}
